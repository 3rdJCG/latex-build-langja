%!TEX encoding = UTF-8
% +++
% latex = "uplatex"
% +++
% --build requirements-------------------
% |	command	: uplatex					|
% |	Fonts	: NotoSansCJKjp				|
% |			  NotoSerifCJKjp			|
% ---------------------------------------
\documentclass[uplatex,dvipdfmx]{jsarticle}

\usepackage[noto]{pxchfon}	%for use Noto fonts
\usepackage{here}			%for use figure here
\usepackage[RPvoltages,americanresistors,americaninductors,europeanvoltage,americancurrents]{circuitikz}

% for use paste soucecode
\usepackage[cache=false]{minted}
\usepackage{listings}
\usepackage{plistings}
\lstnewenvironment{soucecode}[1][]
	{\lstset{
		frame=single,
		basicstyle=\ttfamily,
		numbers=left,
		numbersep=10pt,
		tabsize=2,
		extendedchars=true,
		xleftmargin=17pt,
		framexleftmargin=17pt,
		#1
	}
}{}

\begin{document}

\title{ビルドテスト}
\author{Ibuki KOSHINO}
\maketitle

\section{はじめに}
	この文書はビルドテスト用文書です。

\section{数式のテスト}
	これはオイラーの公式である。
	\begin{equation}
		\mathrm{e}^{\mathrm{i}\theta} = \cos(\theta) + \mathrm{i}\sin(\theta)
	\end{equation}

\section{図のテスト}
	\subsection{回路図のテスト}
		アクティブローパスフィルタの回路図を図\ref{sch:activeLPF}に示す。
		\begin{figure}[H]
			\centering
			\begin{circuitikz}[scale=1.1,/tikz/circuitikz/bipoles/length=1.2cm]
				\draw (0,0)
				node[op amp](OPA){};

				\draw (OPA.out)
				to[short,-] ++(1,0)                     coordinate(Vo)
				to[open,v^=$V_o$,o-o] ++(0,-1)          coordinate(VoGND)
				node[ground](GND){};

				\draw (OPA.-)
				to[short,-] ++(-0.5,0)
				to[R,l_=$R_1$,-] ++(-1,0)
				to[short,-] ++(-0.5,0)                  coordinate(Vi);

				\draw let \p1=(Vi),\p2=(VoGND) in (Vi)
				to[open,v=$V_i$,o-o] ++($(0,\y2)-(0,\y1)$)
				node[ground](GND){};

				\draw let \p1=(OPA.-),\p2=(OPA.out) in (OPA.-)
				to[short,*-] ++(0,1)                    coordinate(overOPAin)
				to[R,l=$R_2$,-] ++($(\x2,0)-(\x1,0)$)   coordinate(overOPAout)
				to[short,-*] (OPA.out);

				\draw let \p1=(overOPAin),\p2=(overOPAout) in (overOPAin)
				to[short,*-] ++(0,1)
				to[C,l=$C$] ++($(\x2,0)-(\x1,0)$)
				to[short,-*] (overOPAout);

				\draw let \p1=(OPA.+),\p2=(VoGND) in (OPA.+)
				to[short,-] ++($(0,\y2)-(0,\y1)$)
				node[ground](GND){};

			\end{circuitikz}
			\caption{アクティブローパスフィルタ}
			\label{sch:activeLPF}
		\end{figure}

	\newpage
	\subsection{ソースコード出力のテスト}
		これはplistingsでのソースコード出力のテストです。
		\begin{soucecode}[language=TeX,caption=基本的な\LaTeX 文書,label=lst:test]
\begin{document}
\section{セクション}
	これはセクションです。

	\subsection{サブセクション}
		これはサブセクションです。

\end{document}
		\end{soucecode}

		これはmintedでのソースコード出力のテストです。
		\begin{minted}[linenos,fontsize=\small,highlightlines={7-16}]{tex}
\documentclass[uplatex,dvipdfmx]{jsarticle}

\title{テスト文書}
\author{わたし}
\date{\today}

\begin{document}
	\maketitle

	\section{セクション}
		セクションです。

		\subsection{サブセクション}
			サブセクションです。

\end{document}
		\end{minted}

\end{document}